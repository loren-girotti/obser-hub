\documentclass{article}

% Language setting
% Replace `english' with e.g. `spanish' to change the document language
\usepackage[spanish]{babel}

% Set page size and margins
% Replace `letterpaper' with `a4paper' for UK/EU standard size
\usepackage[letterpaper,top=2cm,bottom=2cm,left=3cm,right=3cm,marginparwidth=1.75cm]{geometry}

% Useful packages
\usepackage{amsmath}
\usepackage{graphicx}
\usepackage[colorlinks=true, allcolors=blue]{hyperref}

\title{Final Laboratorio de Previsión del Tiempo}
\author{Lorenzo Girotti}
\date{16/05/2025}
\begin{document}
\maketitle
\section{Análisis nacional por regiones Martes 13/05--00Z}
\subsection{Región Noroeste}
La región del NOA presenta un importante gradiente de presión producto de la acción de los anticiclones 
posicionados tanto al este como al oeste del país. Sumado a los efectos de baja presión dinámica al encontrarse a sotavento 
de la cordillera. Cielos de parcial a totalmente cubiertos, con nubosidad baja 
(contrastando con la imagen satelital de topes nubosos).

Vientos leves sin seguir una dirección predominante, quizás acomodando su circulación a las bajas presiones presentes.

\subsection{Región Noreste y Centro}
Sobre el NEA, como en la región pampeana toma acción el anticiclón del atlántico, el cual favorece los cielos despejados y 
vientos calmos o leves. Haciendo foco en la provincia de Bs. As., las condiciones de humedad y presión alta favorecen el 
desarrollo de neblinas.

Los vientos predominan del noreste. Las temperaturas se encuentran alrededor de los 15$^\circ$C.

\subsection{Región Patagónica}
Para la región patagónica tenemos una masa de aire más seca e influenciada por la entrada de un frente frío al norte de Santa Cruz,
lo cual genera nubosidad prefrontal en algunas localidades de Chubut y postfrontal en localidades de Santa Cruz. Para marcar el 
frente me guie de los vientos, las tendencias de presión,  la carta de espesores y la nubosidad vista por imagen satelital.

\subsection{Análisis de cartas de modelo por región.}
\subsubsection{Sistemas de presión a nivel del mar y espesores 1000/500}
Como mencionamos previamente, se posiciona el anticiclón del atlántico con su centro cuya latitud coincide con la del centro de 
la provincia de Buenos Aires. Por el Pacífico hallamos una vaguada al sudoeste de Neuquén, y más al sudoeste hallamos un centro
de baja presión ocluido.
\par Posicionado al sudeste del continente, cercano a la Antártida, encontramos al centro de baja presión asociado al frente frío que
transita la Patagonia.

\subsubsection{Temperatura de rocío y Viento en 850hPa}
Se deja ver la influencia del anticiclón sobre la región pampeana a través de ese 
núcleo de aire seco ubicado al sudeste de la provincia de Buenos Aires. Continuamos la pendiente frontal, aunque
se perturba un poco la discontinuidad de temperatura de rocío por el ingreso de aire húmedo proveniente del
Pacífico en la cordillera neuquina.
\par Sobre el sector norte, predominan vientos del norte de mediana intensidad. Fuerte contraste de humedad en el centro del país.

\subsubsection{Convergencia de humedad 850hPa y HR en 700hPa}
En la carta de convergencia de humedad y temperatura potencial, podemos detectar el gradiente de ésta que sigue la pendiente frontal
sobre la Patagonia. Por otro lado, la HR de 700hPa nos permite visualizar la nubosidad prefrontal que observamos en la imagen satelital.
Recordemos que esa zona está siendo advectada por vientos del oeste, cargados de humedad del Pacífico.

\subsubsection{Temperatura y vientos, geopotencial y vorticidad en 500hPa}
Sobre el sector centro y norte podemos observar vientos predominantes del oeste y un eje de mínimas temperaturas sobre la cordillera,
lo cual genera inestabilidad por advección fría en niveles medios. Sin embargo, donde más se percibe este fenómeno es sobre la zona de 
la cordillera baja (sur de Neuquén y Río Negro, norte de Chubut). Allí estamos ante una delantera de vaguada, y frente a vientos
superiores a los 50kt.
\par Sobre el centro del país podemos observar una pequeña vaguada que presenta perturbaciones de vorticidad en la provincia
de Buenos Aires.

\subsubsection{Vientos máximos y geopotencial en 250hPa}
Podemos identificar al jet subtropical posado al oeste de Mendoza, al sur de éste, hallamos la vaguada junto con un fuerte 
gradiente de geopotencial y vientos máximos (jet polar) corriente arriba de ésta; como también corriente corriente abajo, asociados
a la fuerte zona baroclínica generada por el frente frío mencionado anteriormente.
\par La zona delimitada en celeste se encuentra en la salida polar del jet subtropical como en la entrada ecuatorial del jet streak
polar que se encuentra en la patagonia. Siguiendo un modelo de cuatro cuadrantes, las componentes ageostróficas del viento generan 
en dicha zona divergencia en este nivel; por consecuencia, se favorecen los ascensos en niveles medios y bajos.
\par Esta combinación de patrones y sistemas sobre la región de la Patagonia andina, son propicios para la precipitación intensa.

\section{Pronóstico para la ciudad de La Plata}
\subsection{Martes por la madrugada}
La ciudad se encuentrará bajo la influencia del anticiclón, con vientos del sector norte.
Cielos despejados, temperatura de 12$^\circ$C con una humedad relativa del 90\%. Se esperan neblinas del tipo radiativas.
\begin{itemize}
    \item Cielo despejado.
    \item Viento suave del noreste (6--11km/h).
    \item Temperatura de 12$^\circ$C.
    \item Humedad relativa del 90\%.
    \item Nieblas/neblinas radiativas.
\end{itemize}

\subsection{Martes por la mañana}
Continuan las condiciones de alta presión. Se sostiene el viento suave del noreste. El cielo despejado favorece la incidencia de radiación solar,
mediante la cual comenzaran a disiparse las nieblas/neblinas.
\begin{itemize}
    \item Cielo despejado.
    \item Viento suave del noreste (6--11km/h).
    \item Temperatura de 14$^\circ$C.
    \item Humedad relativa del 85\%.
\end{itemize}

\subsection{Martes por la tarde}
Por la tarde alcanzaremos la temperatura máxima de 24$^\circ$C, lo cual disminuirá significativamente la humedad relativa del ambiente, ya que no ingresará
humedad a la región puesto que los vientos rotarán al norte y aumentarán levemente su intensidad. El anticiclón	conservará su posición, con eso y en vista de
que los niveles bajos se mantendrán secos, la nubosidad se verá inhibida.
\begin{itemize}
    \item Cielo despejado.
    \item Viento leve del norte (12--19km/h).
    \item Temperatura de 24$^\circ$C.
    \item Humedad relativa del 40\%.
\end{itemize}

\subsection{Martes por la noche}
El frente frío avanzará y comenzará a desplazar la cuña que atraviesa la provincia de Buenos Aires. Ingresará aire más húmedo en niveles bajos, y se presentarán
zonas de convergencia de humedad en 850hPa en el sector noreste de la provincia. Esto favorecerá la aparición de nubosidad; sin embargo, las advecciones cálidas y
de vorticidad anticiclónica en niveles medios mantendrá un perfil estable. El viento rotará al noreste y la temperatura descenderá por enfriamiento radiativo.

\begin{itemize}
    \item Cielo ligeramente nublado.
    \item Viento suave del noreste.
    \item Temperatura de 15$^\circ$C.
    \item Humedad relativa de 70\%.
\end{itemize}

\subsection{Resumen de martes}
\begin{itemize}
    \item Temperatura mínima: 12$^\circ$C.
    \item Temperatura máxima: 24$^\circ$C.
    \item Vientos suaves del sector norte.
    \item Cielos despejados a ligeramente nublados.
\end{itemize}


\subsection{Miércoles por la madrugada}
El avance del frente desplazará finalmente al anticiclón como principal sistema de presión sobre la región y provocará que en niveles bajos roten los vientos del
sector norte al sector oeste lo cual indica una advección cálida en la capa por la relación de viento térmico.
\par Junto con el frente, avanzará la vaguada asociada en niveles medios, aunque seguirá manteniendo una advección cálida; por tanto se mantendrá la atmósfera
estable.

\begin{itemize}
    \item Cielo ligeramente nublado.
    \item Temperatura de 11$^\circ$C.
    \item Viento suave del norte.
    \item Humedad relativa del 90\%.
\end{itemize}

\subsection{Miércoles por la mañana}
El frente ingresará al sur de la provincia, lo cual seguirá contribuyendo al ingreso de humedad y formación de nubosidad prefrontal.
\par Los vientos rotarán al sector noroeste.
\par En niveles medios la vaguada asociada al frente provocará perturbaciones de vorticidad relativa sobre la provincia, lo cual comenzará
a inestabilizar lentamente la atmósfera.

\begin{itemize}
    \item Cielo parcialmente nublado.
    \item Viento suave del noroeste.
    \item Temperatura de 15$^\circ$C.
    \item Humedad relativa del 80\%.
\end{itemize}

\subsection{Miércoles por la tarde}
El frente se desplazará al sudeste, dejando trás de sí una zona baroclínica en el centro de la provincia.
El posicionamiento del jet subtropical coincidirá con dicha zona, y comenzará a inestabilizar el centro de la provincia.
\par Los vientos en superficie rotarán al oeste.
\par La convergencia de humedad sobre el sur de Santa Fe y norte de Buenos Aires, favorecerá la formación de nubosidad.

\begin{itemize}
    \item Cielo parcialmente nublado.
    \item Viento suave del oeste.
    \item Temperatura de 20$^\circ$C.
    \item Humedad relativa del 75\%.
\end{itemize}

\subsection{Miércoles por la noche}
La zona baroclínica se debilitará, y los sistemas asociados también. La circulación sobre la provincia será mayoritariamente del oeste 
en todos los niveles; salvo en superficie, donde rota al este permitiendo ingreso de aire húmedo del Río de La Plata.
\par La advección de humedad en niveles bajos y las cálidas en niveles medios generarán una columna de aire cargado de humedad.
Situación propicia para la formación de nubosidad baja estratificada.

\begin{itemize}
    \item Cielo cubierto.
    \item Temperatura de 15$^\circ$C.
    \item Viento suave del este.
    \item Humedad relativa del 75\%.
\end{itemize}

\subsection{Jueves por la mañana}
En el sur, durante el día miércoles, se habrán dado las condiciones para que se dé una ciclogénesis al sur de Tierra del Fuego.
\par El frente frío del ciclón dejará precipitaciones en la patagonia andina y comenzará a inestabilizar la zona central del país.
Aunque más por la noche, puesto que se encontrará lejos todavía.
\par En el sector sur de la provincia, habrá advección de aire frío en niveles medios, aunque no muy intensa, asociada a la vaguada occidental
que se mantendrá durante la semana.
\par Los vientos de niveles altos, favorecerán la divergencia y por tanto los ascensos debajo de ellos, en la zona oeste de la provincia.
\par Sobre el centro de la provincia tendremos precipitaciones que serán disparadas por las perturbaciones de vorticidad ciclónica en niveles medio,
asociadas al ingreso de vientos intensos que pasarán por la cordillera.

Para La Plata:
\begin{itemize}
    \item Cielo cubierto.
    \item Viento suave del noreste.
    \item Temperatura de 17$^\circ$C.
    \item Humedad relativa del 90\%.
\end{itemize}

\subsection{Jueves por la noche}
Luego del pasaje del frente por la Patagonia, y en contraposición al viento del sector norte que persistirá en niveles bajos, las perturbaciones
se intensificarán y debido al alto contenido de humedad, así lo harán las precipitaciones en el sudeste de la provincia. 
\par Esto provoca una atmósfera inestable en la región, lo cual puede propiciar algunas tormentas aisladas en las cercanías de La Plata.

\begin{itemize}
    \item Cielo cubierto.
    \item Viento suave del noreste.
    \item Temperatura de 20$^\circ$C.
    \item Baja probabilidad de lluvias aisladas.
\end{itemize}

\end{document}