\documentclass{article}

% Language setting
% Replace `english' with e.g. `spanish' to change the document language
\usepackage[spanish]{babel}

% Set page size and margins
% Replace `letterpaper' with `a4paper' for UK/EU standard size
\usepackage[letterpaper,top=2cm,bottom=2cm,left=3cm,right=3cm,marginparwidth=1.75cm]{geometry}

% Useful packages
\usepackage{amsmath}
\usepackage{graphicx}
\usepackage[colorlinks=true, allcolors=blue]{hyperref}

\title{Final Laboratorio de Previsión del Tiempo}
\author{Lorenzo Girotti}
\date{16/05/2025}
\begin{document}
\maketitle
\section{Análisis nacional por regiones Martes 13/05-00Z}
\subsection{Región Noroeste}
La región del NOA presenta un importante gradiente de presión producto de la acción de los anticiclones 
posicionados tanto al este como al oeste del país. Sumado a los efectos de baja presión dinámica al encontrarse a sotavento 
de la cordillera. Cielos de parcial a totalmente cubiertos, con nubosidad baja 
(contrastando con la imagen satelital de topes nubosos).

Vientos leves sin seguir una dirección predominante, quizás acomodando su circulación a las bajas presiones presentes.

\subsection{Región Noreste y Centro}
Sobre el NEA, como en la región pampeana toma acción el anticiclón del atlántico, el cual favorece los cielos despejados y 
vientos calmos o leves. Haciendo foco en la provincia de Bs. As., las condiciones de humedad y presión alta favorecen el 
desarrollo de neblinas.

Los vientos predominan del noreste. Las temperaturas se encuentran alrededor de los 15$^\circ$C.

\subsection{Región Patagónica}
Para la región patagónica tenemos una masa de aire más seca e influenciada por la entrada de un frente frío al norte de Santa Cruz,
lo cual genera nubosidad prefrontal en algunas localidades de Chubut y postfrontal en localidades de Santa Cruz. Para marcar el 
frente me guie de los vientos, las tendencias de presión,  la carta de espesores y la nubosidad vista por imagen satelital.

\subsection{Análisis de cartas de modelo por región.}
\subsubsection{Sistemas de presión a nivel del mar y espesores 1000/500}
Como mencionamos previamente, se posiciona el anticiclón del atlántico con su centro cuya latitud coincide con la del centro de 
la provincia de Buenos Aires. Por el Pacífico hallamos una vaguada al sudoeste de Neuquén, y más al sudoeste hallamos un centro
de baja presión ocluido.
\par Posicionado al sudeste del continente, cercano a la Antártida, encontramos al centro de baja presión asociado al frente frío que
transita la Patagonia.

\subsubsection{Temperatura de rocío y Viento en 850hPa}
Se deja ver la influencia del anticiclón sobre la región pampeana a través de ese 
núcleo de aire seco ubicado al sudeste de la provincia de Buenos Aires. Continuamos la pendiente frontal, aunque
se perturba un poco la discontinuidad de temperatura de rocío por el ingreso de aire húmedo proveniente del
Pacífico en la cordillera neuquina.
\par Sobre el sector norte, predominan vientos del norte de mediana intensidad. Fuerte contraste de humedad en el centro del país.

\subsubsection{Convergencia de humedad 850hPa y HR en 700hPa}
En la carta de convergencia de humedad y temperatura potencial, podemos detectar el gradiente de ésta que sigue la pendiente frontal
sobre la Patagonia. Por otro lado, la HR de 700hPa nos permite visualizar la nubosidad prefrontal que observamos en la imagen satelital.
Recordemos que esa zona está siendo advectada por vientos del oeste, cargados de humedad del Pacífico.

\subsubsection{Temperatura y vientos, geopotencial y vorticidad en 500hPa}
Sobre el sector centro y norte podemos observar vientos predominantes del oeste y un eje de mínimas temperaturas sobre la cordillera,
lo cual genera inestabilidad por advección fría en niveles medios. Sin embargo, donde más se percibe este fenómeno es sobre la zona de 
la cordillera baja (sur de Neuquén y Río Negro, norte de Chubut). Allí estamos ante una delantera de vaguada, y frente a vientos
superiores a los 50kt.
\par Sobre el centro del país podemos observar una pequeña vaguada que presenta perturbaciones de vorticidad en la provincia
de Buenos Aires.

\subsubsection{Vientos máximos y geopotencial en 250hPa}
Podemos identificar al jet subtropical posado al oeste de Mendoza, al sur de éste, hallamos la vaguada junto con un fuerte 
gradiente de geopotencial y vientos máximos (jet polar) corriente arriba de ésta; como también corriente corriente abajo, asociados
a la fuerte zona baroclínica generada por el frente frío mencionado anteriormente.
\par La zona delimitada en celeste se encuentra en la salida polar del jet subtropical como en la entrada ecuatorial del jet streak
polar que se encuentra en la patagonia. Siguiendo un modelo de cuatro cuadrantes, las componentes ageostróficas del viento generan 
en dicha zona divergencia en este nivel; por consecuencia, se favorecen los ascensos en niveles medios y bajos.
\par Esta combinación de patrones y sistemas sobre la región de la Patagonia andina, son propicios para la precipitación intensa.



\end{document}