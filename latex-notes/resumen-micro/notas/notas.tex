\documentclass[utf-8,11pt,a4paper]{article}
\usepackage[spanish]{babel}
\usepackage{amsmath,amssymb}
\usepackage{hyperref}
\usepackage{titling}

    

\begin{document}

\begin{titlepage}
    \centering
    {\bfseries\LARGE Universidad Nacional de La Plata \par}\vspace{1cm}
    {\scshape\Large Facultad de Ciencias Astronómicas y Geofíscias \par}\vspace{3cm}
    {\scshape\Huge Micrometeorología \par}\vspace{3cm}
    {\itshape\Large Notas teóricas \par}
    \vfill
    {\Large Autor: \par}
    {\Large Lorenzo Girotti \par}
    \vfill
    {\Large 2025 \par}
\end{titlepage}

\section{Energía}

\subsection{Flujos de energía en una superficie ideal}

Consideramos \emph{superficie ideal} a aquella que es relativamente suave, horizontal, homogénea, extensiva y opaca a la radiación. La energía disponible para tal superficie se simplifica de manera tal que solo depende de flujos verticales de energía.

\subsubsection{Tipos de flujo de energía}



\section{Fludios viscosos, turbulentos}
\subsection{Número de Richardson}
\begin{equation}
    \mathrm{Ri}=\frac{g}{T_{v}}\frac{\partial \Theta_{v}}{\partial z}\left|\frac{\partial V}{\partial z}\right|^{-2}
\end{equation}
\par El número de Richardson es una buena medida de la turbulencia y provee un criterio simple para la existencia o no existencia de turbulencia en un entorno estable estratificado.
\par Un $\mathrm{Ri}>0.25$ indica poco o casi nulo entorno turbulento. Por lo tanto, un perfil vertical de $\mathrm{Ri}$ deja estudiar con más exactitud la turbulencia en la PBL.
\subsection{Número de Reynolds}
\begin{equation}
    \mathrm{Re}=\frac{UL}{\nu}
\end{equation}
donde $U$ es la velocidad característica y $L$ es la longitud de escala.

\section{Definiciones}

\subsection{Viscosidad}

Es una propiedad molecular del fluido que representa la resistencia interna del mismo a la deformación. 
En un fluido \emph{ideal} o \emph{no viscoso} se asume un flujo no turbulento o laminar, en consecuencia no hay transferencias de momento, calor ni masa debido a la mezcla turbulenta; sino que las propiedades son transportadas a lo largo de las lineas de corriente producto de la advección. Por otro lado, la condición de no viscosidad implica que ante la interacción del fluido con una superficie o con otro fluido con una gran diferencia de densidad, no habrá fricción entre ellos.

\subsection{Fluidos Newtonianos}
La viscosidad es responsable de la resistencia friccional entre capas adyacentes de fluido; la resistenciapor unidad de área se llama \emph{tensión por cortante} y se asocia al movimiento relativo entre las capas.

Los fluidos newtonianos son aquellos en donde hay una relación proporcional entre la tensión por cortante y el cambio del gradiente vertical de velocidad. Donde el coeficiente de proporcionalidad $\mu$ se llama viscosidad dinámica del fluido. Para flujos se suele utilizar la viscosidad cinemática que es la viscosidad dinámica dividido la densidad, con dimensiones de $L^2 T^{-1}$

\subsection{Flujos viscosos}

En la realidad siempre existe viscosidad en los flujos. Aún así podemos encontrar circunstancias en donde el flujo se vuelva \emph{laminar}.

\subsubsection{Flujo laminar}

Se caracteriza por ser suave, ordenado y de movimiento lento, donde las capas adyasentes de fluidose deslizan entre sí con muy poca transferencia (solo a nivelmolecular) de propiedades a través de ellas. El campo de flujo, la temperatura asociada y los campos de concetración son regulares y predecibles y solo varían gradualmente en tiempo y espacio.

\subsubsection{Flujo turbulento}

Son movimientos altamente irregualres, casi aleatorios, tridimensionales, altamente rotantes, disipativos y muy difusos (mezcla). Todas las propiedades escalares y las del flujo fluctuan tanto en tiempo como en espacio, con un amplio rango temporal y espacial. Por ejemplo: las fluctuaciones de velocidad en la ABL van desde $10^{-3}\mathrm{s}$ a $10^4\mathrm{s}$ y la correspondiente al espacio va desde $10^{-3}\mathrm{m}$ a $10^4\mathrm{m}$ - del orden del millón en rango. Es por esto que es imposible predecir o calcular exactamente a los movimientos turbulentos como funciones del tiempo y el espacio; normalmente se utilizan los promedios estadísticos de las propiedades.



\end{document}