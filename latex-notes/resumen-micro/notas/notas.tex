\documentclass[utf-8,11pt,a4paper]{article}
\usepackage[spanish]{babel}
\usepackage{amsmath,amssymb}
\usepackage{hyperref}

\begin{document}
\section{Fludios viscosos, turbulentos}
\subsection{Número de Richardson}
\begin{equation}
    \mathrm{Ri}=\frac{g}{T_{v}}\frac{\partial \Theta_{v}}{\partial z}\left|\frac{\partial V}{\partial z}\right|^{-2}
\end{equation}
\par El número de Richardson es una buena medida de la turbulencia y provee un criterio simple para la existencia o no existencia de turbulencia en un entorno estable estratificado.
\par Un $\mathrm{Ri}>0.25$ indica poco o casi nulo entorno turbulento. Por lo tanto, un perfil vertical de $\mathrm{Ri}$ deja estudiar con más exactitud la turbulencia en la PBL.
\subsection{Número de Reynolds}
\begin{equation}
    \mathrm{Re}=\frac{UL}{\nu}
\end{equation}
donde $U$ es la velocidad característica y $L$ es la longitud de escala.

\end{document}